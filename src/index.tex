\section{Aula 6.1 - Códigos de linha}

O objetivo dessa vídeo aula é mostrar os diversos esquemas de codificação existentes na modulação digital da informação.
O ponto crucial para entender a distinção entre esquemas é compreender como os esquemas são classficados visto que entendendo a "lógica de classificação",
fica fácil mapear um esquema ao resultado final de uma sequência de bits passando pelo mesmo. Para facilitar a compreensão introduziremos cada categoria 1 a 1 e terminaremos
com um apanhado contrastando o uso de cada uma.

\subsection{NRZ - Não Retorno ao Zero}

Porque o nomo NRZ, o que significa não retornar a zero? Veja a imagem abaixo para tentar associar a explicação ao resultado real.

\includegraphics[width=0.6\textwidth]{../assets/nrz.png}

Bem o não retorno ao zero está no fato que o sinal elétrico de saída, o resultado da codificação, oscila sempre entre nos limites do conjunto [ +V,-V ] consequentemente nunca o bit0/1 assume o valor
de 0V, daí o nome, não retorno ao zero.

vamos agora as subidivisões que existem

\subsubsection{As possíveis subdivisões, ao que você deve estar atento?}

\textbf{L} (level) (do inglês nível): nessa subdivisão existe um mapeamento claro entre o nível lógico 0 e -V e nível lógico 1 e +V, é a figura de cima, o ponto crucial é que essa subdivisão
não é afetada pelo valor do bit anterior, não existe memória na codificação, o mapeamento é estático.
\\

\textbf{M/S} (mark/space) (do inglês marca/espaço): nessas subdivisões existe um nível lógico que alterna e outro que fica fixo e nível de tensão igual ao do que alterou, logo um nível lógico
marca a tensão usada pelo outro, efetivamente, no caso \textbf{M} é para o caso que o bit 1 é o que marca e \textbf{S} é o caso que o bit 0 é marcado.

\includegraphics[width=0.6\textwidth]{../assets/nrzm.png}

\textbf{OBS:}Note que o 1 é o nível lógico que se altera nesse caso.

\textbf{OBS2:}Note que a imposição de um nível lógico ser alterado se da pelo fato que se não alterasse seria impossível distinguir os níveis lógicos visto a regra de codificação
mencionada previamente.

\subsection{RZ - Retorno ao zero}

O detalhe aqui é perceber que o nível lógico, indepedente da subdivisão, \textit{sempre retorna para o 0 por metade do intervalo de codificação de um bit}.

\includegraphics[width=0.6\textwidth]{../assets/rz.png}


\subsubsection{As possíveis subdivisões, ao que você deve estar atento?}

\textbf{Polar}: a distinção é clara, os níveis lógicos no final do período de transição são necessariamente +V ou -V, muito parecido com o NRZ-L, mas note que há esse retorno ao
zero que da o nome a essa família de esquemas.
\\
\\
\includegraphics[width=0.6\textwidth]{../assets/polar-rz.png}
\\
\textbf{OBS:} Nesse caso é convenção que o bit 1 seja codificado por tensão positiva e o bit 0 negativa.
\\

\textbf{AMI} (Alternate mark inversion): o nome leva marca, mas note que pela representação abaixo do esquema não existe marca, visto que o bit 0 é sempre 0, infelizmente
é uma inconsistência de nomeclatura, \textbf{note que na imagem o autor não compensou o offset inicial da codificação, mas usualmente o offset é o mesmo que da anterior}.
o ponto é que a codificaçao é curiosa visto que o bit 1 pode ser codificado tanto como +V e -V é típico ter problemas de sincronismo com esse esquema se vários zeros
forem enviados em sequência.
\\

\includegraphics[width=0.6\textwidth]{../assets/ami.png}

\subsection{Bifase}

Aqui que a imaginação precisa ser um pouco maior, os níveis deixaram de ser tão óbvios, não é mais zero no meio de uma transição ou uma transição instantânea como no NRZ
é o tipo da transição, se sobe ou desce, é difícil visualizar isso então deixe eu fornecer uma maneira de pensar a respeito da distinção:
\\
\\
\textbf{Se for transição de subida
	imagine que é equivalente a jogar a parte antes do meio da transição para cima efetivamente parecendo com o NRZ-L para nível lógico 1 e se for descida repitir o processo obtendo
	o mapeamento no NRZ para nível lógico 0, conclusão, segundo essa lógica o bit 0 do NRZ seria mapeado no bit 1 do bifase e o contrário para o bit 1 do NRZ}.
\\

\textit{OBS:note ainda que as transições intermediárias, ocorrem no meio da transição, são garantidas por esse esquema, mas a que ocorrem no final não}
\\

\includegraphics[width=0.6\textwidth]{../assets/bifasel.png}
\\\\
no caso o esquema acima se chama \textit{Manchester ou bifase-L}

Bem ainda existe o \textbf{Manchester Diferencial-M}, mas esse não será comentado nessas notas.


\section{Aula 6.2 - Modulação M-PAM}

O objetivo desta aula é demonstrar uma técnica alternativa de modulação de sinal digital a modulação M-PAM, aqui constrastaremos a mesma com a alternativa PCM, conceitualmente
o que a difere da previamente mencionada e porque ou não usar uma ou a outra.
\\

\includegraphics[width=0.7\textwidth]{../assets/mpam.png}

\subsection{O que é a modulação PCM mesmo?}

Lembrando do diagrama de comunicação:


\includegraphics[width=0.6\textwidth]{../assets/diagramacomm.png}


Note que o sinal é amostrado isto é a uma dada frequência o valor atual do sinal é capturado e enviado para um conversor ADC,analógico digital, na saída obtemos o sinal quantizado
que é pegar os possíveis valores do sinal analógico, os limites, e associar uma sequência de bits que melhor mapeie aquele nível lógico, note que quanto mais bits mais preciso,
porém mais caro, depois é a etapa de codificação que faz o mapeamento daquela sequência de bits em uma segunda sequência de bits denominada \textbf{símbolos} e agora é onde entra
a modulação PCM, cada símbolo é composto de k bits a modulação PCM envia k bits numa transição de maneira individualizada, cada bit é enviado em frações do período para transição
de forma que é necessário examinar k bits para saber o símbolo transmitido.

\subsection{Porque então a modulação M-PAM?}

Note que há uma escolha na etapa de modulação, é possível fazer um terceiro mapeamento dos k bits em níveis discretos de amplitude, então suponha que um exemplo de k bits
fosse 101 isso seria mapeado no nível de amplitude 5, visto que 000 seria mapeado no 0 de nível de amplitude e 111 seria mapeado em 7 de nível de amplitude, logo diminuindo
o número de bits transmitido efetivamente diminui a largura de banda e essa é a maior vantagem dessa modulação, note no entanto que na velocidade de informação é essencialmente
a mesma visto que tal modulação não comprime o tempo de envio e portanto o tempo para chegar os k bits que representam o símbolo na modulação PCM é o tempo para chegar o sinal
modulado em amplitude.

\subsection{Comparando as abordagens, não há bala de prata \emoji{😢}}

E novamente voltamos a velha tautologia conhecida pelos engenheiros, não há solução perfeita. Ora o que a modulação M-PAM nos fornece é ótimo, mas o gasto de potência é muito maior
visto que não só o transmissor como o receptor precisam modular níveis de amplitudes para o total de símbolos que existam no esquema de comunicação, mas ainda há um outro problema
quanto mais níveis de amplitude maior se torna o problema de detecção simbólica visto que a diferença de níveis de amplitude tende a diminuir e consequenmente erros ficam mais custosos de serem detectados ou até mesmo corrigidos.

\subsection{E não tem equação nessa aula?}

Sim tem uma bastante importante que ajuda a dimensionar a quantidade de bits de acordo com o erro de quantização desejado, lembre-se que indepedente do número de estados para
quantizar a amostra o erro entre estados é no máximo, pior caso, $|e|_{max}=\frac{q}{2}$ e sabendo disso e da seguinte desigualdade: $|e| \le pV_{pp}$ obtemos a relação final:

\begin{equation}
	L \ge \frac{1}{2p}
\end{equation}

\begin{equation}
	l \ge \log_{2}(\frac{1}{2p})
\end{equation}

onde $L=2^{l}$





\section{Aula 6.3 - Multiplexação por divisão no tempo I}

Nessa aula é introduzido o importantíssimo conceito de multiplexação e a multiplexação de divisão no tempo o que são frames e time slots
\\\\
\includegraphics[width=0.6\textwidth]{../assets/muxte.png}

\subsection{O que é multiplexar?}

A idea é que uma interface, no contexto de comunicações geralmente um meio físico como fio, é conectada a múltiplas fontes de sinal e escolhe qual sinal deseja transmitir
baseado em algum critério.


\subsection{Como conceitualmente ocorre a multiplexação de divisão no tempo}

\subsubsection{O que é TDM}
A TDM(time divison multiplexing) ou multiplexação de divisão no tempo é o critério de tempo isto é a banda de transmissão é particionada para alocar diferentes fontes de sinal.
No nosso contexto o sinal vem da entidade tributário que paga para uma companhia telefônica para ter acesso a rede.

\subsubsection{O que são frames e time slots?}

\includegraphics[width=0.4\textwidth]{../assets/tsl.png}
\\\\
Na figura acima fica evidente que há uma certa hierarquia conceitual na TDM os frames são blocos de dados com parcelas dos dados produzidos por cada tributário que é enviado num certo período
os time slots são justamente essa parcela de dados previamente mencionada, note que tecnicamente o sistema de comunicação poderia permitir que apenas um tributário usasse a rede por vez
isto é ficar alternando entre tributários por algum critério, mas aí a taxa de envio percebida por um tributário seria abismal essa tradeoff do TDM permite que todos os tributários tenham uma experiência
previsível e justa, visto que a cada, segundo a imagem, 3T s se passam cada tributário enviou parte de seus dados, enquanto que no outro caso seria 3T*(\#tributarios-1) ou seja
linear a complexidade computacional.

\subsection{O bit de sincronização}

Bem para o receptor conseguir decodificar corretamente a menssagem é importante garantir que o pacote recebido, o frame, seja de fato o que se espera e não houve
corrupções no envio para isso é adotado um bit de sincronização que segue um padrão interno que consegue validar a ordem de recebimento de pacotes e garantir a leitura correta
dos bits enviados por cada tributário.





\section{Aula 6.4 - Multiplexação por divisão no tempo II}

Nessa aula ficamos a par do padrão internacional T-1.

\subsection{O padrão T-1}

Esse padrão foi desenvolvido pela Bell Labs, empresa americana, e é um exemplo prático de como o TDM é aplicado na realidade.
a composição da hierarquia T-1 é dada da seguinte maneira:
\\\\
\textbf{24 canais de voz -> para cada canal um modulador PCM -> mux}
\\
A largura de banda de cada canal é de \textit{4Khz} e a taxa de bits que é enviada é de \textit{64kpbs}

E o modulador PCM amostra a uma taxa de \textit{8k amostras/s} e \textit{8 bits para etapa de quantização}

E por fim cada linha fornece \textit{8 bits+1 de sincronismo} para compor o frame

E a taxa de frames é de \textit{8k frames/s}

\includegraphics[width=0.7\textwidth]{../assets/tdm.png}

\subsection{A hierarquia TDM}

E para terminar essa seção a saída do mux é cascateada em outro mux segundo uma sequência de multiplicadores até que chegue no nível mais elevado que em tese
seriam as taxas de enlace mais rápidas fornecidas pelo sistema. Abaixo uma comparação entre modelo Europeu e Americano.

\includegraphics[width=0.8\textwidth]{../assets/hire.png}

Note que os números na saída dos blocos são o número de cascatas que ocorrem antes de chegar no próximo nível da hierarquia, note ainda que quanto maior o número a
direita da letra E4 por exemplo indica uma maior taxa de enlace e note que há uma comunicação entre o bloco branco, USA, cinza escuro, Japão e o que sobrou Europa num
determinado nível da hierarquia

